\apendice{Plan de Proyecto Software}

\section{Introducción}

A lo largo de este apéndice se tratará todo aquello relacionado con la planificación del proyecto, considerándose esta un punto clave para cualquier desarrollo de \textit{software}. En esta fase se estima tanto el dinero, como el trabajo y el tiempo que se va a emplear en completar el proyecto a través de un análisis minucioso de los recursos necesarios.

La fase de planificación se encuentra dividida en:

\begin{itemize}
	\item Planificación temporal.
	\item Estudio de viabilidad.
	\begin{itemize}
		\item Viabilidad económica.
		\item Viabilidad legal.
	\end{itemize}
\end{itemize}

\section{Planificación temporal}

En esta sección se elaborará un programa de tiempos en los que se estima la duración de cada una de las partes del proyecto. Partiendo del establecimiento de una fecha de inicio y una fecha de finalización estimada, a través tanto del peso de cada una de las tareas como de los requisitos necesarios para poder empezar cada una.

Al inicio del proyecto se decidió utilizar \textit{Scrum} como metodología ágil para la gestión del proyecto. Debido a que el equipo formado no contaba con más de cuatro personas, no se ha podido seguir a rajatabla, pero sí que se han seguido las líneas generales de esta filosofía:

\begin{itemize}
	\item Estrategia de desarrollo incremental a través de \textit{sprints} (o iteraciones) y revisiones.
	\item La duración media de cada \textit{sprint} era de dos semanas.
	\item Al finalizar cada \textit{sprint} se realizaban reuniones en las que se revisaba el incremento en el proyecto y se planificaba el siguiente \textit{sprint}.
	\item Tras la planificación del \textit{sprint} se creaban una serie de tareas a realizar, las cuales eran estimadas y priorizadas en un tablero \textit{canvas}.
	\item La monitorización del progreso del proyecto se llevó a cabo a través de gráficos \textit{burndown}.
\end{itemize}

Cabe comentar que la estimación se realizó a través de \textit{story points} (incluidos por ZenHub) que tienen una traducción temporal mostrada en la Tabla A.1.

\tablaSmall{Equivalencia entre \textit{story points} y tiempo.}{l l}{equivalenciaEstimacion}{Story points & Estimación temporal \\}{
1 & 30'\\
2 & 2h\\
3 & 3h\\
5 & 5h\\
8 & 8h\\
13 & 13h\\
21 & 21h\\
40 & 40h\\
}

La estimación de \textit{story points} asignada a cada tarea no tiene porque ser el tiempo real que se ha tardado en resolver. Algunas tareas llevaron más tiempo y otras menos, pero normalmente se trataba de estimar haciendo que sobrara tiempo para no pillarse los dedos. 

\textit{Sprints} llevados a cabo:

\subsection{Sprint \#0 (01/10/2018 - 14/10/2018)}

A lo largo de este \textit{sprint} pre-inicial se realizó una lectura e investigación sobre los diferentes Trabajos de Fin de Grado disponibles  y se tuvo la primera toma de contacto con la plantilla \LaTeX  que se está utilizando para realizar esta documentación. Aún no se controlaba bien la asignación de tareas ni la estimación de tiempos, como se puede ver en la Figura A.1 o en enlace a el \href{https://github.com/FranBurgos/TFG/milestone/1?closed=1}{\textit{sprint} \#0}.

\imagen{sprint0}{\textit{Sprint} \#0.}

\subsection{Sprint \#1 (14/10/2018 - 27/10/2018)}

En este \textit{sprint} acontece el primer contacto con los doctores Merino y Sarabia. Se comenzó con una presentación de los materiales que se iban a utilizar en el proyecto, expresando las primeras ideas y conjeturas que cada uno tenía sobre el mismo. También se definieron el tipo de repositorio a utilizar, así como la metodología y el formato de la documentación. Tras la reunión se decidió que la próxima semana se dedicaría a la investigación sobre el lenguaje a utilizar para el desarrollo del proyecto, buscando facilitar la comunicación serie con la placa NXP FRDM K64F. En la Figura A.2 se puede observar el avance sobre el \href{https://github.com/FranBurgos/TFG/milestone/2?closed=1}{\textit{sprint} \#1}.  A este \textit{sprint} se dedicaron 13 \textit{story points}, lo que se traduciría a 13 horas reales.

\imagen{sprint1}{\textit{Sprint} \#1.}

Debido a las dudas que se presentaban al principio de este proyecto tuvo lugar una segunda reunión en la que se abordaron los siguientes temas:
\begin{itemize}
	\item Préstamo de la placa FRDM K64F al estudiante.
	\item Conexión de la placa mediante puerto serie y visualización de los datos a través del programa Termite.
	\item Explicación del documento ``Especificaciones Interfaz planta piloto.docx".
	\item Pasos siguientes:
	\begin{itemize}
		\item Crear una aplicación sencilla que se comunique con la placa a través del puerto serie en C++.
		\item Definir un formato del archivo de configuración, donde se guardan los valores de las variables que se van a utilizar en la placa. La aplicación no tiene porqué trabajar siempre con las mismas variables, depende de cómo se configure la placa, por lo que deberá permitir introducir las variables que el usuario considere y trabajar con ellas.
	\end{itemize}
	\item Se decide utilizar Visual Studio 2015 como IDE por mejor usabilidad con C++ y Windows Form.
\end{itemize}

\subsection{Sprint \#2 (28/10/2018 - 10/11/2018)}

Durante estas dos semanas, y debido a que la elección primera del lenguaje de desarrollo iba a ser C++, se realizaron una serie de tutoriales sobre dicho lenguaje y sobre su uso en Visual Studio 2017. En la Figura A.3 se puede observar el avance sobre el \href{https://github.com/FranBurgos/TFG/milestone/3?closed=1}{\textit{sprint} \#2}. A este \textit{sprint} se dedicaron 82 \textit{story points}, lo que se traduciría a 82 horas reales.

\imagen{sprint2}{\textit{Sprint} \#2.}

Tras realizar varias investigaciones sobre la comunicación serie en otros lenguajes se tomó la decisión de cambiar de C++ a C\# en el lenguaje en el que se iba a desarrollar la herramienta, puesto que ofrecía un gran abanico de facilidades y se contaba con mayor experiencia de uso en este segundo lenguaje.

\subsection{Sprint \#3 (10/11/2018 - 24/11/2018)}

En este periodo se comenzó a programar de una manera más eficiente, consiguiendo notables avances sobre el código y a su vez, comunicando la aplicación con la placa NXP. A su vez, se diseñó un icono para personalizar la aplicación y el formato en el que las configuraciones que iban a ser cargadas en la aplicación se guardaban en un archivo de texto. En la Figura A.4 se puede observar el avance sobre el \href{https://github.com/FranBurgos/TFG/milestone/4?closed=1}{\textit{sprint} \#3}. A este \textit{sprint} se dedicaron 29 \textit{story points}, lo que se traduciría a 29 horas reales.

\imagen{sprint3}{\textit{Sprint} \#3.}

Cabe añadir que a partir de este momento se tuvo presente la adición de varios idiomas a la aplicación para dotarla de mayor apertura cultural.

\subsection{Sprint \#4 (24/11/2018 - 08/12/2018)}

A lo largo de este \textit{sprint} se llevaron acabo diversas mejoras en la interfaz de la aplicación y en el tratamiento de datos recibidos a través del puerto serie. No obstante, se podría definir como la principal mejora la implementación de una base de datos, con su correspondiente investigación previa, y la conexión de la misma a Visual Studio y al proyecto. 

El método utilizado para la conexión fue creado a partir de un asistente del IDE mencionado, el cual facilitó esta implementación, pero a su vez creó el mayor quebradero de cabeza que ha aparecido durante todo el desarrollo. En el momento de querer ejecutar la aplicación en otro ordenador, y, debido a haber creado la conexión de manera local en mi equipo, la aplicación no conseguía funcionar correctamente. Este problema continuó hasta el último \textit{sprint}. En la Figura A.5 se puede observar el avance sobre el \href{https://github.com/FranBurgos/TFG/milestone/5?closed=1}{\textit{sprint} \#4}. A este \textit{sprint} se dedicaron 44 \textit{story points}, lo que se traduciría a 44 horas reales.

\imagen{sprint4}{\textit{Sprint} \#4.}

\subsection{Sprint \#5 (08/12/2018 - 22/12/2018)}

Durante este \textit{sprint} se mostró a los tutores una primera versión de la interfaz de la aplicación, quienes reportaron una serie de mejoras que fueron añadidas al \textit{sprint} que estamos tratando, como podrían ser el archivo de ayuda en la aplicación, etiquetas faltantes o funcionalidades no tenidas en cuenta (por ejemplo, que el usuario pudiera definir la cantidad de datos mostrados en la gráfica en tiempo real). A lo largo de este periodo se libera la primera \textit{pre-release V0.0} de la aplicación. En la Figura A.6 se puede observar el avance sobre el \href{https://github.com/FranBurgos/TFG/milestone/6?closed=1}{\textit{sprint} \#5}. A este \textit{sprint} se dedicaron 103 \textit{story points}, lo que se traduciría a 103 horas reales.

\imagen{sprint5}{\textit{Sprint} \#5.}

\subsection{Sprint \#6 (22/12/2018 - 05/01/2019)}

En este \textit{sprint} se lleva a cabo una reunión en la que se incluye al Dr. López Nozal, quien aporta nuevas ideas para mejorar tanto el código como la gestión de proyecto, como por ejemplo, la inclusión de la herramienta Codacy en el mismo. También se plantea la posibilidad de añadir un archivo de log a la aplicación, haciendo más fácil la detección de errores. En la Figura A.7 se puede observar el avance sobre el \href{https://github.com/FranBurgos/TFG/milestone/7?closed=1}{\textit{sprint} \#6}. A este \textit{sprint} se dedicaron 54 \textit{story points}, lo que se traduciría a 54 horas reales.

\imagen{sprint6}{\textit{Sprint} \#6.}

En este momento se vaticina cuáles van a ser las funciones que se va a poder cubrir a lo largo de este proyecto y comienzan a barajarse ideas de trabajo futuras. 

\subsection{Sprint \#7 (05/01/2019 - 19/01/2019)}

A lo largo de este \textit{sprint} y tras reunirme con los tutores y evaluar la \textit{pre-release V0.1} se definen los siguientes pasos a seguir en el proyecto. La mayor parte del tiempo invertido en este \textit{sprint} está dedicado a resolver las puntualizaciones que los tutores comentaron sobre dicha \textit{pre-release}, y que se puede encontrar en el \textit{issue} \href{https://github.com/franburgos/tfg/issues/57}{57}. En la Figura A.8 se puede observar el avance sobre el \href{https://github.com/FranBurgos/TFG/milestone/8?closed=1}{\textit{sprint} \#7}. A este \textit{sprint} se dedicaron 28 \textit{story points}, lo que se traduciría a 28 horas reales.

\imagen{sprint7}{\textit{Sprint} \#7.}

\subsection{Sprint \#8 (19/01/2019 - 02/02/2019)}

Durante este \textit{sprint} se trabaja sobre la mejora de pequeños detalles de la aplicación, así como el problema que se comentó en el \textit{sprint \#4} sobre los problemas que se encontraban al ejecutar la aplicación en otro ordenador. Tras una ardua investigación se consiguió disipar la duda y solventar dicho error, pudiendo así testear la aplicación sin problema. En consecuencia a esta investigación se liberaron diferentes \textit{releases} en las que se añadió la creación de la base de datos desde código y a partir de las cuales se pudieran hacer pruebas. En la Figura A.9 se puede observar el avance sobre el \href{https://github.com/FranBurgos/TFG/milestone/9?closed=1}{\textit{sprint} \#8}. A este \textit{sprint} se dedicaron 38 \textit{story points}, lo que se traduciría a 38 horas reales.

\imagen{sprint8}{\textit{Sprint} \#8.}

\subsection{Sprint \#9 (02/02/2019 - 14/02/2019)}

Este \textit{sprint} se ha dedicado en su totalidad a la documentación del proyecto y a preparar los archivos y ejecutables necesarios para concluir el desarrollo del proyecto. En la Figura A.10 se puede observar el avance sobre el \href{https://github.com/FranBurgos/TFG/milestone/10?closed=1}{\textit{sprint} \#9}. A este \textit{sprint} se dedicaron 80 \textit{story points}, lo que se traduciría a 80 horas reales.

\imagen{sprint9}{\textit{Sprint} \#9.}

\subsection{Seguimiento de velocidad}

Gracias a la herramienta ZenHub se puede obtener un gráfico con el que se informa de la velocidad que se ha tenido en cada \textit{sprint} del proyecto, aportándonos una media de trabajo y una imagen global de nuestros avances, como podemos ver en la Figura A.11.

\imagen{velTrack}{Seguimiento de velocidad.}

\section{Estudio de viabilidad}

\subsection{Viabilidad económica}

En el siguiente apartado se analizarán los costes y beneficios que podría haber supuesto el proyecto en el caso de que se hubiese realizado en un entorno empresarial real.

\subsubsection{Coste de personal}

El proyecto se lleva a cabo por un desarrollador junior empleado a tiempo parcial (30 horas semanales) durante cuatro meses, ergo se considera el siguiente salario \cite{web:irpf} \cite{web:irpf2}

\tablaSmallSinColores{Coste de personal.}{l r}{costePersonal}{Concepto & Coste \\}{
Salario neto anual & 14.968,30\euro\\
Retención IRPF anual (10,49\%) & 1.888,7\euro\\
Seguridad social anual (28,30\%) & 1.143,00\euro\\
Salario bruto anual & 18.000,00\euro\\
\textbf{Total (4 meses} & \textbf{6.000,00\euro}\\
}

\subsubsection{Costes de material}

El material utilizado para el proyecto ha sido un ordenador valorado en 700\euro, del cual se pondrá el coste amortizado a cuatro años, y una placa NXP FRDM K64F valorada en 50\euro, durante 4 meses:

\tablaSmallSinColores{Coste de material.}{l l r}{costeMaterial}{Concepto & Coste inicial & Coste amortizado(\euro) \\}{
Ordenador & 700,00\euro & 58,33\euro\\
NXP FRDM K64F & 50,00\euro & 50,00\euro\\
\textbf{Total} & & \textbf{108,33\euro}\\
}

\subsubsection{Costes de \textit{software}}

El software utilizado para el proyecto, aplicando la amortización por el tiempo usado (4 meses), ha sido el siguiente:

\tablaSmallSinColores{Coste de \textit{software}.}{l l r}{costeSoft}{Concepto & Coste inicial & Coste amortizado(\euro) \\}{
Microsoft Windows 10 Pro & 259,00\euro & 21,58\euro\\
Microsoft Visual Studio 2017 & 45,00\euro mes & 160,00\euro\\
Microsoft SQL Server & 900,00\euro & 75,00\euro\\
\textbf{Total} & & \textbf{256,58\euro}\\
}

\subsubsection{Costes totales}

El sumatorio de todos los costes ha sido el siguiente:

\tablaSmallSinColores{Coste totales.}{l r}{costeTot}{Concepto & Coste (\euro) \\}{
Personal & 6.000,00\euro\\
Materiales & 108,33\euro\\
\textit{Software} & 256,98\euro\\
\textbf{Total} & \textbf{6365,31\euro}\\
}

\subsubsection{Beneficios}

Debido a que la finalidad de esta aplicación es la utilización por parte de los alumnos de Grado en Ingeniería Electrónica y Automática de la Universidad de Burgos, no se considera en ningún momento obtener beneficio alguno, siempre y cuando la institución no quiera otorgar cierta cuantía por los servicios prestado, en cuyo caso se aceptarán gustosos.

\subsection{Viabilidad legal}

En este apartado se realiza un estudio sobre las leyes vigentes que puedan afectar al proyecto desarrollado y cómo se solucionarán los posibles problemas que surjan.

Gracias a haber desarrollado toda la aplicación con software de Microsoft, del cual ya se han pagado las licencias pertinentes, no es necesario atender a la utilización de más licencias de terceros.

Añadiendo a lo anterior, cabe mencionar que al tratarse de una aplicación con finalidad no comercial, las restricciones sobre el software creado son mucho menores.

Concluyendo, la licencia que asignaré a esta aplicación será una licencia Creative Commons ``Reconocimiento-NoComercial-CompartirIgual 4.0 Internacional'', como se puede ver en la Figura A.12 \cite{web:cc}.

\imagen{licencia}{Licencia Creative Commons.}

