\capitulo{3}{Conceptos teóricos}

En este capítulo se expondrán los conceptos teóricos que se van a ver reflejados en la aplicación, permitiendo al lector tener una base de conocimiento en la que apoyarse para la comprensión del proyecto.

\section{.NET Framework}

\href{https://dotnet.microsoft.com/}{.NET} es un framework de Microsoft que hace un énfasis en la transparencia de redes, con independencia de plataforma de hardware y que permite un veloz desarrollo de aplicaciones. Basada en ella, la empresa pretende desarrollar una estrategia horizontal que integre todos sus productos, desde el sistema operativo hasta las herramientas de mercado \cite{wiki:dotNet}.

.NET Framework es una tecnología que admite la compilación y ejecución de la última generación de aplicaciones y Servicios web XML. El diseño de .NET Framework está enfocado a cumplir los siguientes objetivos:
\begin{itemize}
	\item Proporcionar un entorno de ejecución de código que promueve la ejecución segura del mismo y que reduzca todo lo posible los conflictos de versiones y la implementación de software.
	\item Proporcionar un entorno coherente de programación orientada a objetos.
	\item Fomentar la integración del código de .NET con otros tipos de código basando la comunicación en estándares del sector.
	\item Ofrecer al programador coherencia entre tipos de aplicaciones muy diferentes, como las basadas en Windows o en Web.
\end{itemize}

.NET Framework tiene dos componentes principales: la biblioteca de clases de .NET Framework y Common Language Runtime (CLR). En la siguiente ilustración se puede apreciar la relación de Common Language Runtime y la biblioteca de clases con el sistema en su conjunto y las aplicaciones. En la ilustración se representa igualmente cómo funciona el código administrado dentro de una arquitectura mayor \cite{web:docNet}.

\imagen{docNetContext}{.Net Context}

\subsection{Biblioteca de clases de .Net Framework}

La biblioteca de clases de .NET Framework es una colección de tipos reutilizables que se integran estrechamente con Common Language Runtime. Al ser una biblioteca de clases orientada a objetos, proporciona tipos de los que su propio código administrado deriva funciones. Esto hace que los tipos de .NET Framework sean fáciles de usar, reduciendo así el tiempo asociado con el aprendizaje de las nuevas características de .NET Framework. Cabe añadir que los componentes de terceros se integran fácilmente con las clases de .NET Framework.

\subsection{Common Language Runtime (CLR)}

Common Language Runtime gestiona la memoria, la ejecución de código, la ejecución de subprocesos, la comprobación de la seguridad del código, la compilación y el resto servicios del sistema. Estas características son intrínsecas del código administrado que se ejecuta en Common Language Runtime.

\section{SQL}



\section{Comunicación Serie}
\section{Placa NXP FRDMK64F}

En este capítulo se expondrán los conceptos teóricos con los que se ha trabajado en el desarrollo del proyecto.

\section{Secciones}

Las secciones se incluyen con el comando section.

\subsection{Subsecciones}

Además de secciones tenemos subsecciones.

\subsubsection{Subsubsecciones}

Y subsecciones. 


\section{Referencias}

Las referencias se incluyen en el texto usando cite \cite{wiki:latex}. Para citar webs, artículos o libros \cite{koza92}.


\section{Imágenes}

Se pueden incluir imágenes con los comandos standard de \LaTeX, pero esta plantilla dispone de comandos propios como por ejemplo el siguiente:

\imagen{escudoInfor}{Autómata para una expresión vacía}



\section{Listas de items}

Existen tres posibilidades:

\begin{itemize}
	\item primer item.
	\item segundo item.
\end{itemize}

\begin{enumerate}
	\item primer item.
	\item segundo item.
\end{enumerate}

\begin{description}
	\item[Primer item] más información sobre el primer item.
	\item[Segundo item] más información sobre el segundo item.
\end{description}
	
\begin{itemize}
\item 
\end{itemize}

\section{Tablas}

Igualmente se pueden usar los comandos específicos de \LaTeX o bien usar alguno de los comandos de la plantilla.

\tablaSmall{Herramientas y tecnologías utilizadas en cada parte del proyecto}{l c c c c}{herramientasportipodeuso}
{ \multicolumn{1}{l}{Herramientas} & App AngularJS & API REST & BD & Memoria \\}{ 
HTML5 & X & & &\\
CSS3 & X & & &\\
BOOTSTRAP & X & & &\\
JavaScript & X & & &\\
AngularJS & X & & &\\
Bower & X & & &\\
PHP & & X & &\\
Karma + Jasmine & X & & &\\
Slim framework & & X & &\\
Idiorm & & X & &\\
Composer & & X & &\\
JSON & X & X & &\\
PhpStorm & X & X & &\\
MySQL & & & X &\\
PhpMyAdmin & & & X &\\
Git + BitBucket & X & X & X & X\\
Mik\TeX{} & & & & X\\
\TeX{}Maker & & & & X\\
Astah & & & & X\\
Balsamiq Mockups & X & & &\\
VersionOne & X & X & X & X\\
} 
