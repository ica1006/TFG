\capitulo{4}{Técnicas y herramientas}

En este apartado se detallarán las técnicas y herramientas utilizadas para desarrollar el proyecto.

\subsection{4.1. Metodología}

\subsubsection{4.1.1. Scrum}
Scrum es un marco de trabajo para el desarrollo de \textit{software} que se engloba dentro de las metodologías ágiles. Sus principales características son la adopción de una estrategia de desarrollo incremental, basar la calidad del resultado en el conocimiento tácito de los integrantes de equipos auto organizados, y en el solapamiento de las diferentes fases del desarrollo \cite{wiki:scrum}.

En este caso los \textit{sprints} han sido de dos semanas, tras los cuales se realizaba una reunión para definir las nuevas tareas del siguiente \textit{sprint} y realizar una retroalimentación del \textit{sprint} recién hecho.

\subsubsection{4.1.2. Técnica de Pomodoro}

Para mejorar la concentración durante el desarrollo del proyecto se ha utilizado la técnica Pomodoro, consistente en intervalos de 25 minutos de concentración intensa seguidos de descansos de 5 minutos, y descansos de 30 minutos después de haber completado cuatro ciclos 25+5.\cite{wiki:pomodoro}

\subsection{4.2. Patrones de diseño}

\subsubsection{4.2.1. MVC - Modelo Vista Controlador}

A la hora de estructurar el desarrollo de la aplicación se ha optado por la utilización del Modelo Vista Controlador, consiguiendo así separar la capa que representa la realidad, la capa que conoce los métodos y atributos del modelo, recibiendo y realizando las peticiones del usuario, y la capa visible para el usuario \cite{web:patronDis}.

\subsection{4.3. Control de versiones}

a\textbf{ZenHub}

\subsection{4.4. Entorno de desarrollo integrado (IDE)}

a

\subsection{4.5. Comunicación}

a

\subsection{4.6. Documentación}

aa

\subsection{4.7. Librerías}

a

\subsection{4.8. Otras herramientas}

aa